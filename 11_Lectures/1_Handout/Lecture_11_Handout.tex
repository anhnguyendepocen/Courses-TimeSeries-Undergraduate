\documentclass[12] {article}
\usepackage{setspace}
\usepackage{amssymb,amsmath}


\begin{document}
\onehalfspace

\title{Lecture 11}
\date{}
\maketitle

\section{Estimation in an MA(1) model: $\theta_1$ and $\sigma^2$}

Consider the MA(1) model:

\[ X_t =  \mu + \epsilon_{t} + \theta_1 \epsilon_{t-1}, \quad E[\epsilon_t]=0 \quad E[\epsilon_t^2]=\sigma^2. \]

\noindent We are trying to figure out how to use time series data $(X_1, X_2, \ldots X_{T})$ to estimate the parameters $\mu, \theta_1, \sigma^2$. \\

For $\mu$ (which is the population mean of $X_t$) we have already discussed the properties of the estimator: 

\[ \widehat{\mu} = \frac{1}{T} \sum_{t=1}^{T} X_t. \]

\noindent We also know that:

\[\gamma_0 = \sigma^2 + \theta_1^2 \sigma^2, \quad \gamma_1 = \theta_1 \sigma^2. \]

This gives $\theta_1$ and $\sigma^2$ as a function of  $\gamma_0$ and $\gamma_1$ which have direct counterparts in the data (denote them $\widehat{\gamma}_0, \widehat{\gamma}_1$). Thus, one way to try to figure out what are the values of $\theta_1$ and $\sigma^2$ is to solve the system of equations:
\begin{eqnarray}
\gamma_0 &=&  \sigma^2 + \theta_1^2 \sigma^2 \\
\gamma_1 &=& \theta_1 \sigma^2  
\end{eqnarray}

\noindent  From (2) we get that $\theta_1 = \gamma_1/\sigma^2$ and from (1), we get that:
 
\[ \gamma_0 =  \sigma^2 + \frac{\gamma_1^2}{\sigma^2} \]

\noindent Multiplying the expression by $\sigma^2$ we have the expression:

\[ (\sigma^2)^2 -\gamma_0 \sigma^2 +  \gamma_1^2 \]

\noindent which is a quadratic equation in $\sigma^2$. Solving for $\sigma^2$ we get 

\[\sigma^2 = \frac{\gamma_0 \pm \sqrt{\gamma_0^2 - 4 \gamma_1^2}}{2},\]

\noindent which we can further re-write as:

\[ \sigma^2 = \gamma_0 \left( \frac{1 \pm \sqrt{1 - 4 (\gamma_1/\gamma_0)^2}}{2} \right) \]

\noindent and 

\[ \theta_1 = \frac{\gamma_1/\gamma_0}{\left( \frac{1 \pm \sqrt{1 - 4 (\gamma_1/\gamma_0)^2}}{2} \right)} \]

\noindent A natural estimator for $\sigma^2$ and $\theta_1$ replaces $\gamma_0$ and $\gamma_1$ by $\widehat{\gamma}_0$ and $\widehat{\gamma}_1$. Note that we are finding two solutions, but only  one them satisfies $|\theta_1| < 1$.\\









 








 
 

\end{document}